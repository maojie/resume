\documentclass[11pt, a4paper]{moderncv}

% moderncv theme.
\moderncvtheme[green]{classic}

% character encoding.
\usepackage[utf8]{inputenc}

% adjust the page margins.
\usepackage[scale=0.75]{geometry}
\recomputelengths  % required when changes are made to page layout lengths

% personal information.
\firstname{Jeremy}
\familyname{Mao}
\title{Curriculum Vitae}
\mobile{+86 18621008390}
\email{maojie0924@gmail.com}

\nopagenumbers{}

\begin{document}

\maketitle

\section{Education}
\cventry{2007-2011}{East China Normal University}{Software Institute}{Software Engineering}{}{}
\cventry{}{Grade Rank: 15\%}{Awarded scholarships several times}{}{}{}

\section{Work Experience}
\cventry{2014-present}
{Intel Collaboration Suite for WebRTC}
{C++, Java, JavaScript}
{Product}{}
{
Intel CS for WebRTC is a high-performance and cross-platform web real-time communication solution. I was responsible for the development of the Android client SDK, designed and implemented the hardware-accelerated H.264/VP8 support at both client and server sides which reduced the cpu usage and the power consumption to make the streaming video achieve the best user experience across different Intel platforms.
}

\vspace*{0.2\baselineskip}
\cventry{2012-2014}
{Chromium/WebKit/WebRTC}
{C++, Java, Python}
{Open Source Project}{}
{
Designed and implemented the getUserMedia feature based on the W3C spec in Chromium M15 firstly before Chrome officially supports WebRTC. Contributed several features including the getUserMedia infobar, WebRTC support for x86 Android, hardware-accelerated VP8 support in Chrome for Android and ChromeOS for IA etc. and also fixed lots of bugs in Chromium, WebKit and WebRTC. Due to the contribution, I became a committer in WebRTC project in 2013.
}

\vspace*{0.2\baselineskip}
\cventry{2011-2012}
{Kona}
{C++, Java}
{Research Project}{}
{
Kona is a research project which aims to replace the Dalvik runtime in Android with the standard Java runtime to make the converted Android apps work like standard Java apps on MeeGo. I was responsible for analyzing the hotspot Android APIs which were used to develop tons of Android apps, designed and implemented several Android apps to test the converting function of eclipse KDT tool and the runtime behavior.
}

\section{Internship}
\cventry{2010}{Summer Internship at Intel Developer Relations Division}{}{}{}{Designed and implemented the High Availability plugin to manage the virtual machines in the cloud cluster and also implemented the corresponding Java APIs based on Libvirt to manage the virtual machines.}
\cventry{2011}{Internship at Intel Software Optimization Technology Center}{}{}{}{Designed and implemented the daily build and nightly test framework for Kona project.}

%\section{Work Experience}
%\cventry{2011-present}{Software Engineer at Intel Web Technology Organization}{}{}{}{Focus on the WebRTC technology in Chromium Browser especially for x86 Android, designed and implemented several key features and optimizations to make WebRTC achieve the best user experience across different Intel platforms.}

\section{Skills}
\cventry{Languages}{C++ > Java > Python}{}{}{}{}
\cventry{OS}{Linux, Android, Mac}{}{}{}{}
\cventry{Open Source}{Chromium, WebKit, WebRTC}{}{}{}{}
%\cventry{English}{Passed CET-6, fluent in oral and written English}{}{}{}{}

\section{Awards}
\cventry{2008}{National ITAT Educational Programming Cup Third Prize}{}{}{}{}
\cventry{2011}{Intel Excellent Intern of the Year}{}{}{}{}
\cventry{2012, 2013}{Intel Division Recognition Award x 4}{}{}{}{}
\cventry{2013}{Nominated as Intel Employee of the Year}{}{}{}{}
\cventry{2014}{Intel Group Recognition Award}{}{}{}{}

\section{Community}
\cventry{Blog}{\url{maojie.github.io}}{}{}{}{}
\cventry{Github}{\url{github.com/maojie}}{}{}{}{}

\end{document}